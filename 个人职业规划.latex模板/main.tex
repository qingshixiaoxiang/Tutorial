\documentclass{article}
\usepackage[UTF8]{ctex}
\usepackage{geometry}
\usepackage{multirow}
\usepackage{natbib}
\geometry{left=3.18cm,right=3.18cm,top=2.54cm,bottom=2.54cm}
\usepackage{graphicx}
\pagestyle{plain}	
\usepackage{setspace}
\usepackage{enumerate}
\usepackage{caption2}
\usepackage{datetime} %日期
\renewcommand{\today}{\number\year 年 \number\month 月 \number\day 日}
\renewcommand{\captionlabelfont}{\small}
\renewcommand{\captionfont}{\small}
\begin{document}

\begin{figure}
    \centering
    \includegraphics[width=8cm]{upc.png}

    \label{figupc}
\end{figure}

	\begin{center}
		\quad \\
		\quad \\
		\heiti \fontsize{45}{17} \quad \quad \quad 
		\vskip 1.5cm
		\heiti \zihao{2} 《计算科学导论》个人职业规划
	\end{center}
	\vskip 2.0cm
		
	\begin{quotation}
% 	\begin{center}
		\doublespacing
		
        \zihao{4}\par\setlength\parindent{7em}
		\quad 

		学生姓名:\underline{\qquad  吴晓云 \qquad \qquad}

		学\hspace{0.61cm} 号:\underline{\qquad 1907010201\qquad}
		
		专业班级:\underline{\qquad 计科1902 \qquad  }
		
        学\hspace{0.61cm} 院:\underline{计算机科学与技术学院}
% 	\end{center}
		\vskip 1.5cm
		\centering
		\begin{table}[h]
            \centering 
            \zihao{4}
            \begin{tabular}{|c|c|c|c|c|c|c|c|c|}
            % 这里的rl 与表格对应可以看到,姓名是r,右对齐的;学号是l,左对齐的;若想居中,使用c关键字。
                \hline
                \multicolumn{5}{|c|}{分项评价} &\multicolumn{2}{c|}{整体评价}  & 总    分 & 评 阅 教 师\\
                \hline
                自我 & 环境 & 职业 & 实施 & 评估与 & 完整性 & 可行性 &\multirow{2}*{} &\multirow{2}*{}\\
                分析& 分析& 定位 & 方案 & 调整 & 20\% & 20\% & ~&~ \\\            
                10\% & 10\% & 15\% & 15\% & 10\% & &  &~ &~\\
                \cline{1-7} 
                & & & & & & & ~&~ \\
                & & & & & & & ~&~ \\
                \hline      
            \end{tabular}
        \end{table}
		\vskip 2cm
		\today
	\end{quotation}

\thispagestyle{empty}
\newpage
\setcounter{page}{1}
% 在这之前是封面,在这之后是正文
\section{自我分析}
	
	\par
	
\subsection{自然条件}\par
性别:女\par
年龄:19岁\par
身体状况:身体健康、无不良嗜好。\par
居住地:山东省潍坊市昌乐县\par

\subsection{性格分析}
\par
平时比较安静,遇到自己感兴趣的事,会坚持的做下去。但若是遇到很不感兴趣的,会变得非常被动。\par
\subsection{教育与学习经历}
\par
小学:鄌郚中心小学\par
初中:鄌郚中学\par
高中:山东省昌乐及第中学\par
大学:中国石油大学(华东)\par

\subsection{工作与社会阅历}
\par
工作经验:曾找过一份暑假工为2个月的服务员。\par
社会阅历:高三毕业后曾去北京待过一个月,见过很多不同的人,特别是外国的人,文化不同。\par

\subsection{知识、技能与经验}
\par
如今的知识水平是高中的知识加上大一一学期所学的内容,学习了一门程序设计语言C++,没有工作中运用的经验,只是平时做网站上的题目。\par
\subsection{兴趣爱好与特长}
\par
从高中开始,我的兴趣爱好就是计算机,写程序,但是因为高中时期着重于文化课的学习,只能把我的兴趣搁置,但是大学我考上了我想学的专业,开始了我的兴趣爱好。我没有艺术方面的特长,现在正在努力培养某一专业学科方面的特长。\par
\section{环境分析}

\subsection{社会环境分析}
\par
总体来说,我们现在面临一个非常好的宏观环境,社会安定,政治稳定,经济发展迅速,并与全球一体化接轨,法制建设不断完善,文化繁荣自由,尖端技术、高新技术突飞猛进。\par
经济形势:“全面小康”将在2020年实现,预示经济增长将维持总体平稳。“十三五”规划将在2020年收官,重点领域投资还将加快。“三大攻坚战”进入最后一年,财政与货币政策将保持适度宽松。2019年前三季度,计算机、电子行业的投资增速达11.6%,
远高于制造业整体仅2.5%
的增速水平。\par
就业形势:据我国权威部门预测表明,随着我国经济与社会的发展,科学技术的进步,今后几年对专门人才的需求将有较大的变化。急需高新技术人才、信息技术人才等。表明如今计算机科学与技术专业人员就业形势很好。\par

\subsection{家庭环境分析}
\par
未婚、出生于普通农民家庭、家人希望毕业后找一份好的工作。\par
我处在一个农村家庭,父母受教育水平相对较低,他们以后对于我的工作城市会有要求,不想让我离家太远,最好是在本省,但是很明显,想要发展得更好,在大城市的IT公司是很重要的,也是我梦寐以求的。\par

\subsection{职业环境分析}
\par
行业现状:计算机技术在不断发展中,其广泛应用在政府、教育、 农业等诸多领域。家用电脑也开始得到普及,其应用的广泛性 更是推动了信息技术的发展进程。目前,计算机在诸多领域的 应用前景非常可观。如,网络工程专业的学生在毕业后,可以 从事大型的电商服务或通信设备制造等领域的工作,也可以到 一些事业单位从事网络技术的维护或培训工作。而软件工程专 业的学生毕业后,则可以从事国家机关或学校的技术开发、软 件开发等工作。总之,目前的计算机科学技术的应用领域正在 不断扩大。\par
发展前景:(1)计算机科学技术应用将更加普遍。\par
(2)计算机科学技术将会向人性化发展。\par
(3)计算机科学技术智能化发展。\par
工作内容:\par
软件方面:\par 
1、参与软件工程系统的设计、开发、测试等过程;\par
2、协助工程管理人保证项目的质量;负责工程中主要功能的代码实现;\par
3、解决工程中的关键问题和技术难题;\par
4、协调各个程序员的工作,并能与其它软件工程师协作工作。还要编写各种各样的软件说明书,如:需求说明书,概要说明书等。
\par
编程方面:\par
1、负责软件项目的详细设计、编码和内部测试的组织实施,对小型软件项目兼任系统分析工作,完成分配项目的实施和技术支持工作。\par
2、协助项目经理和相关人员同客户进行沟通,保持良好的客户关系。\par
3、参与需求调研、项目可行性分析、技术可行性分析和需求分析。\par
4、熟悉并熟练掌握交付软件部开发的软件项目的相关软件技术。
\par
5、负责向项目经理及时反馈软件开发中的情况,并根据实际情况提出改进建议。\par
6、参与软件开发和维护过程中重大技术问题的解决,参与软件首次安装调试、数据割接、用户培训和项目推广。\par
7、负责相关技术文档的拟订。\par
8、负责对业务领域内的技术发展动态进行分析研究。\par
算法方面:\par
1.     负责相关产品关键算法的仿真和开发实现;\par
2.     负责新算法的调研和仿真验证;\par
3.     负责算法的实现,配合FPGA或DSP完成相关工作的开发;\par
4.     撰写算法设计开发文档。\par



\subsection{地域与人际环境分析}
\par
气候水土:气候适宜,冬暖夏凉。\par
文化特点:既有很多的古建筑,有一定的古风韵,有极具现代化,科技发达。\par
发展前景:发展前景广阔,尤其是科技方面发展迅速,有很大的发展空间。\par
人脉与人际关系:有分散在全国各地的同学。特别是大城市,比如北京,上海等等。

\par 

\section{职业定位}

\par
职业定位包括:我在高一时就已经大体知道了自己想要从事的职业,并不断为其做铺垫,就是算法工程师,当然,更希望是虚实现实研发领域的算法工程师。现在对我来说还很难,我在为之努力,比如,加入ACM,学习算法,熟悉算法。\par

\subsection{行业领域定位与理由}
\par
行业领域定位:虚拟现实终端的研发、虚拟现实内容平台的建设。
\par
理由:自己的兴趣占绝大部分原因,还有一部分是现在这一领域的发展还不成熟,有很大的发展空间。
\par
\subsection{职业岗位起点定位与理由}
\par
职业岗位起点:先在其他的网络公司中做一段时间的算法工作。\par
理由:积累经验,虚拟现实领域的行业要求比较苛刻,工作也不容易。先在其他互联网公司中做算法有关的工作,等积累一部分经验,学习了更多的东西之后,再转去从事虚拟现实领域的工作。\par

\subsection{职业目标与可行性分析}
\par
 
\begin{enumerate}[(1)]
	\item 短期目标(大学4年):掌握基本编程语言,比如C++、python、java等。具有创新意识与科学的思考方法。具备大学生的基本素养。\par
	\item 中长期目标(5-10年):熟练掌握各种算法;可以在虚拟现实领域做算法工程师;有一定的经济能力。\par
\end{enumerate}



\section{实施方案}
\par
\begin{enumerate}[1、]
	\item 在ACM中努力坚持下来,充分利用其所给的资源及比赛经历提升自己。在课程之外,多泡图书馆,利用图书馆中的书。\par
	\item 多待在图书馆和自习室里,多读书,读算法编程方面的书,多练习算法题,逼自己做不想做的事,比如说早起。在学习之外,多锻炼提高身体素质,为以后做打算。认真学习各科课程。英语也很重要,每天学一点英语,比如说背一部分单词。\par
	\item 多交与以后职业相关的朋友,从比自己优秀的人身上学到东西,比如品质,比如知识经验。尽量不与人交恶。\par
	\item 我想要的是先立业后成家,在事业没有进展时一般不考虑成家。我会看中工作,但是,工作不是生活的全部,在努力工作时,也会享受生活。要可持续性发展,保持积极性。\par
	\item 当工作压力太大时,我会出去旅游,抽出一两天,抛开工作,旅游放松心情,看看世界的美好,保证身心健康,心情愉悦,对生活充满希望。回来继续工作。当然,在平时,每天锻炼一下,保证每天精神饱满。\par
\end{enumerate}
\par 

\section{评估与调整}
\par 
\subsection{评估时间}
每学年一次\par
\subsection{评估内容}
总结已达到的成果与已拥有的经济能力,看是否达到自己制定目标中所写的那样,还有职务,距离自己的目标还差多远,综合分析,若与自己制定的目标相差甚远,分析原因,从自身与环境分析,找到问题所在,及时调整。\par
\subsection{调整原则}
看是否距离设立的目标越来越近,若与当时制定的目标渐行渐远,分析原因,看是否是自己的问题,若是自己的问题,从自身着手,找到当时自己存在的问题,并及时调整,端正心态。若制定的目标不符合当时的环境,再分析一次所处环境,然后在根据自己的实际情况调整目标,不要制定大而空的目标,所调整的目标应该能够实现。\par




\end{document}
